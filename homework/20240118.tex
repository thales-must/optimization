\documentclass{article}
\usepackage[utf8]{inputenc}
\title{Engineering Optimization Homework}
\author{Tai Jiang}
\date{January 2024}
\usepackage{amsmath}
\usepackage{amsthm}
\usepackage{amssymb}
\usepackage{tikz}
\usepackage{color}
\usepackage{listings}


\begin{document}
  \maketitle
  \section{Is the intersection of two convex sets convex?}
  \begin{itemize}
    \item lf the intersection is empty, or consists of a single point, it is true bydefinition
    \item Otherwise, take any two points A, B in the intersection. The line ABjoining these points must also lie wholly within their intersection
    \item Therefore, the intersection is a convex set
    \item The intersection of any number of convex sets is also convex
  \end{itemize}

  \begin{proof}
  Let $A$ and $B$ be two convex sets. Show by $A \cap B$ is also convex.

  Consider any two points $x$ and $y$,  $\forall x, y \in A \cap B$.
  
  $x \in A, x \in B, y \in A, y \in B$

  To prove that $A \cap B$ is convex, we need to show that for any $\lambda \in [0,1]$, the point $\lambda x + (1-\lambda)y$ is also in $A \cap B$.

  Since $x,y \in A$, and $A$ is convex, we have $\lambda x + (1-\lambda)y \in A (\lambda \in [0,1])$.

  Similarly, since $x,y \in B$, and $B$ is convex, we have $\lambda x + (1-\lambda)y \in B (\lambda \in [0,1])$.

  Therefore, $\lambda x + (1-\lambda)y$ is in both $A$ and $B$, which implies that $\lambda x + (1-\lambda)y \in A \cap B$.

  Since this holds for any two points $x$ and $y$ in $A \cap B$, it follows that $A \cap B$ is convex.

  Thus, we have proven that the intersection of two convex sets is convex.

  \end{proof}

\end{document}